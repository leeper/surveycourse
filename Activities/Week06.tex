\documentclass[12pt, a4]{article}
\usepackage[top=2cm, bottom=2cm, left=2cm, right=2cm]{geometry}
\usepackage{setspace}
\usepackage{mdwlist}
\usepackage{titlesec}

{\renewcommand{\arraystretch}{1.2}


%opening
\title{Survey Budgeting\vspace{-2em}}
\author{}
\date{}

\begin{document}

\maketitle

\noindent The purpose of this activity is to think about the costs of implementing a survey and the impact of alternative survey procedures on the costs of a survey data collection and the quality of data derived from that design. Up to this point, we have focused on obtaining a particular sample size to obtain population estimates with a desired level of precision. Now we face ``reality constraints,'' where those statistical objectives must be realized using real time, money, and materials. In this task, you will think about your individual exam project and the costs associated with your proposed design. The activity has two parts. First, you will itemize the fixed and variable costs of your project and use these to estimate the total cost of your project. Second, you will consider some particular features of your survey plan and estimate potential cost increases or decreases associated with changing those features.

\section{Budget for Fixed and Variables Costs}

\noindent Your task is to itemize the fixed and variable costs of your project. \textit{Fixed costs} exist regardless of your sample size, but might vary according to how you plan to implement the survey For example, there are different fixed costs associated with a telephone survey, which purchase and use of physical telephones, and an internet survey, which does not require that infrastructure. \textit{Variable costs} depend on the sample size. These are the costs associated with each survey interview and the costs of attempting to complete a survey interview. Additionally, you might have further variable costs that do not increase at a per-interview rate (e.g., such as contingencies for replacing broken computers or telephones, or emergency funds in the event of lost or stolen data or materials).

\subsection{Fixed Costs}

\noindent Estimate total fixed costs for the project.

\begin{itemize}
\item Sampling plan: 
\item Staff salaries (hours spent on project multiplied by wage):
\item Data collection materials (computers, handhelds, telephones):
\item Questionnaire preparation:
\item Data cleaning and analysis:
\item Other fixed costs:\\
\end{itemize}
\textbf{Total fixed costs:}

\subsection{Variable Costs}

\noindent Estimate costs per \textit{interview attempt}. This means the cost of attempting (but not completing) an interview. These costs might include the salary to pay an in-person interviewer to travel to a residence, telephone charges per minute, etc.

\begin{itemize}
\item Interviewer salaries (amount of time to attempt contact multiplied by wage):
\item Interviewer or respondent travel:
\item Other per-attempt costs: \\
\end{itemize}
\textbf{Total per-attempt costs:}

\vspace{2em}

\noindent Now estimate costs per \textit{completed response}. These are in addition to the costs from above of attempting an interview.

\begin{itemize}
\item Interviewer salaries (amount of time to complete interview multiplied by wage):
\item Materials:
\item Interviewer or respondent travel:
\item Incentives paid to respondents:
\item Other per-interview costs:\\
\end{itemize}
\textbf{Total per-interview costs:}

\vspace{2em}

\subsection{Incidental Costs}

\noindent Itemize and estimate the cost of any incidental costs (e.g., contingencies for lost or stolen materials). Incidental costs may not actually be realized, but you need to budget for them in case they occur otherwise the cost of these features has to be made up by taking resources away from other parts of the survey (e.g., reducing the sample size and thus reducing precision of survey estimates).

\begin{itemize}
\item Item and cost: 
\item Item and cost: 
\item Item and cost: 
\end{itemize}
\textbf{Total incidental costs:}


\clearpage
\subsection{Total Survey Budget}

\noindent The total budget for your survey is the sum of the fixed costs, variable costs, and (possibly) the incidental costs. To estimate the total variable costs, you will need to know your intended sample size ($n$) and estimate the response rate (RR), which is the number of completed interviews per attempted interviews.\\


\begin{tabular}{p{1.5in} p{.75in} p{.75in} p{.75in} p{1.5in}}
	\hline\hline
	Item                                 & Cost & Sample Size & Response\newline Rate & Subtotal \\ \hline\hline
	Fixed costs                          &      & --          & --    &  \\ \hline
	Per-attempt                          &      &             &      &  \\ \hline
	Per-interview                        &      &             & --    &  \\ \hline\hline
	Budget Total\newline w/o incidentals &      & --          & --    &  \\ \hline\hline
	Incidental costs                     &      & --          & --    &  \\ \hline
	Budget Total\newline w/ incidentals  &      & --          & --    &  \\ \hline\hline
\end{tabular}

\vspace{1em}
Subtotal per-attempt costs are calculated as:
\begin{equation}
\frac{1}{RR} * n * Costs_{attempt}
\end{equation}

Subtotal per-interview costs are calculated as:
\begin{equation}
n * Costs_{interview}
\end{equation}


\section{Budget for Alternative Designs}

\noindent For each of the aspects of your survey design, consider how it impacts the total costs of the survey.

\begin{enumerate}\itemsep2em
\item Salary for staff is 125\% (i.e., 25\% higher) than what you initially estimated.
\item The response rate is 20 percentage points lower than what you initially estimated. I.e., if you estimated $RR = .6$, it is now $RR = .4$.
\item Respondents in your survey will now be incentivized a higher amount. Add 50 kr. or 10 Euro to the per-interview costs.
\item Due to a change in sampling design, the necessary sample size for your study is now 10\% larger than you estimated.
\item A mode change has been called for. Half of your respondents will complete the interview by telephone and half will complete the study via a web survey.
\end{enumerate}

\end{document}
